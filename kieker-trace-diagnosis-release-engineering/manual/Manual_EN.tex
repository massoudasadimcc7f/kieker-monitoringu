\documentclass{article}

\usepackage[utf8]{inputenc}
\usepackage[english]{babel}
\usepackage[parfill]{parskip}
\usepackage{graphicx}
\usepackage{hyperref}

\begin{document}

  \newcommand{\version}[0]{2.0.0-SNAPSHOT}
  \newcommand{\KiekerTraceDiagnosis}[0]{\texttt{Kieker Trace Diagnosis}}
  \newcommand{\file}[1]{\textit{#1}}

  \title{Kieker Trace Diagnosis\\User Manual}
  \date{Version \version{}}
  \author{Kieker Project\\(kieker-monitoring.net)}

  \maketitle

  \section{Requirements and Installation}
  The following requirements are necessary to use \KiekerTraceDiagnosis{}.
  \begin{itemize}
    \item Windows or Linux
    \item Java Runtime Environment 8 Update 60 or higher
  \end{itemize}
  The usage of Oracle's JDK for the tool is recommended. The troubleshooting for the usage of OpenJDK can be found in section~\ref{OpenJDK}.\\

  For the installation, it is only necessary to unpack the archive \file{Kieker Trace Diagnosis-\version{}-linux.tgz}, respectively \file{Kieker Trace Diagnosis-\version{}-windows.zip}, into an abritary folder.
  The application can subsequently be started with the star script \file{start.sh}, respectively \file{start.bat}, in the folder \file{bin}.
  The application should have write permissions for the folder from which it has been started.

  \section{Licence}
  \KiekerTraceDiagnosis{} is licensed under the Apache License Version 2.0.
  The complete license text can be found in the provided file \file{LICENSE}.
  The licenses of the uses libraries can be found in the \file{.LICENSE} files in the \file{lib} folder.

  \section{Ansichten}

  \subsection{Öffne Monitoring Log}
  Mit einem Klick auf diesen Menüpunkt öffnet sich ein Dialog, mit welchem das zu öffnende Monitoring Log ausgewählt wird.
  Die Analyse beginnt direkt nach Auswahl des Logs.
  Dabei werden alle Monitoring Daten in dem Verzeichnis, sowie in allen Unterverzeichnissen analysiert.

  \subsection{Einstellungen}
  Das Fenster \textit{Einstellungen} ermöglicht die Konfiguration des Werkzeuges.

  \paragraph{Zeiteinheit}
  Hier kann eingestellt werden, in welcher Zeiteinheit die Dauer von Methodenaufrufen und ähnlichem angezeigt werden.
  Dies ist beispielsweise dann sinnvoll, wenn Daten mit Auflösung im Nanosekunden-Bereich gesammelt wurden, die interessanten Methodenaufrufe sich aber eher im Bereich von einigen Millisekunden befinden.

  \paragraph{Operationen}
  Hier kann eingestellt werden, in welcher Form Operationen dargestellt werden.
  Zur Auswahl sind eine verkürzte und eine ausführliche Darstellung.

  \paragraph{Komponenten}
  Hier kann eingestellt werden, in welcher Form Komponenten dargestellt werden.
  Zur Auswahl sind eine verkürzte und eine ausführliche Darstellung.

  \paragraph{Zeitstempel}
  Hier kann eingestellt werden, in welcher Form Zeitstempel von Methodenaufrufen und ähnlichem angezeigt werden.
  Zur Auswahl sind diverse Zeit- und Datumsformate.

  \paragraph{Methodenaufrufe zusammenfassen}
  Hier kann eingestellt werden, ob unsignifikante Methodenaufrufe in Traces zusammengefasst werden sollen.
  Das is beispielsweise dann sinnvoll, wenn Traces mit einigen tausend Methodenaufrufen betrachtet werden, von denen die meisten sehr schnell abgearbeitet wurden und uninteressant sind.  
  Methoden unterhalb des gegebenen Schwellwertes werden zusammengefasst, sodass nur die signifikanteren und länger andauernden Methoden noch sichtbar sind.

  \paragraph{Nicht gemonitorte Zeit anzeigen}
  Hier kann eingestellt werden, ob die Zeiten innerhalb eines Traces, die nicht weiter durch Monitoring-Daten bestimmt werden können, dargestellt werden.
  Angezeigt wird dann die Zeit, welche entweder in der Methode selbst verbraucht wird oder in weiteren Methoden, die nicht gemonitort worden sind. 

  \paragraph{Erlaube reguläre Ausdrücke (Java-Standard)}
  Mit dieser Einstellungen können in den Filtermasken reguläre Ausdrücke nach Java-Standard eingegeben werden.
  Wenn diese Option deaktiviert ist, findet eine Volltextsuche des eingegebenen Textes für das jeweilige Element statt.

  \paragraph{Groß-/Kleinschreibung beachten}
  Hier kann eingestellt werden, ob Groß- und Kleinschreibung in den Filtermasken beachtet wird oder nicht.

  \paragraph{Im gesamten Trace suchen} 
  Standardmäßig werden die Filter für die Trace-Ansichten nur auf den obersten Methodenaufruf angewendet.
  Wenn diese Option aktiv ist, beziehen sich die Filter auf die Methoden in dem gesamten Trace.

  \paragraph{Aktiviere weitere Überprüfungen während der Trace Rekonstruktion}
  Wenn diese Option aktiv ist, werden bei der Trace Rekonstruktion weitere Prüfungen ausgeüfhrt.
  So wird zusätzlich geprüft, ob die Monitoring-Daten konsistent sind.
  Unter Umständen werden so weitere Traces als ungültig markiert und nicht in die weitere Analyse mit aufgenommen.

  \paragraph{Prozentuale Berechnung auf den obersten Methodenaufruf beziehen}
  Standardmäßig bezieht sich die prozentuale Berechnung in den Traces auf den übergeordneten Knoten.
  Wenn diese Option aktiv ist, so bezieht sich die prozentuale Berechnung stattdessen auf den obersten Knoten des Traces, also auf die Einstiegsmethode.

  \subsection{Traces}
  Die Ansicht \textit{Traces} zeigt alle gültigen Traces einzeln an.
  Nicht rekonstruierbare Traces werden nicht angezeigt.

  \subsection{Aggregierte Traces}
  Die Ansicht \textit{Aggregierte Traces} zeigt alle gültigen Traces in ignorierter Form an.
  Nicht rekonstruierbare Traces werden nicht angezeigt.
  Aggregiert werden alle Traces, bei denen der Ausführungspfad identisch ist und die jeweiligen Methoden (im Bezug auf den Ausführungscontainer, die Komponente, die Operation und - falls vorhanden - die Ausnahme) identisch sind.


  \subsection{Methodenaufrufe}
  Die Ansicht \textit{Methodenaufrufe} zeigt alle Methodenaufrufe von gültigen Traces einzeln an.
  Ignorierte und verwaiste Records, sowie Records, die zu nicht rekonstruierbaren Traces gehören, werden nicht angezeigt (siehe auch Abschnitt~\ref{MonitoringLogStatistiken}).
  Ein Doppelklick auf einen Methodenaufruf zeigt den Methodenaufruf in dem zugehörigen Trace an.

  \subsection{Aggregierte Methodenaufrufe}
  Die Ansicht \textit{Aggregierte Methodenaufrufe} zeigt alle Methodenaufrufe von gültigen Traces in aggregierter Form an.
  Ignorierte und verwaiste Records, sowie Records, die zu nicht rekonstruierbaren Traces gehören, werden nicht angezeigt (siehe auch Abschnitt~\ref{MonitoringLogStatistiken}).
  Aggregiert werden alle Methodenaufrufe, bei denen der Ausführungscontainer, die Komponente, die Operation und - falls vorhanden - die Ausnahme identisch sind.
  Ein Doppelklick auf einen aggregierten Methodenaufruf zeigt die aggregierten Methodenaufrufe einzeln an.
  
  \subsection{Monitoring Log Statistiken}\label{MonitoringLogStatistiken}
  Die Ansicht \textit{Monitoring Log Statistiken} zeigt Informationen zu den aktuell geladenen Monitoring-Daten an.
  So lässt sich dort etwa die Dateigröße und Analysedauer entnehmen.
  Ebenso listet die Ansicht allgemeine Statistiken auf, wie etwa die Anzahl der Methodenaufrufe.\\

  Die Zeile \textit{Verwaiste Records} zeigt an, wieviele Records gefunden wurden, die keinem Trace zugeordnet werden konnten.
  Die Zeile \textit{Ignorierte Records} zeigt an, wieviele Records für die Analyse ignoriert wurden.
  Dies sind üblicherweise Records, die für \KiekerTraceDiagnosis{} irrelevant sind, wie beispielsweise Records zum Speicherverbrauch.

  \section{Troubleshooting}

  \subsection{Usage of OpenJDK}\label{OpenJDK}
  The usage of OpenJDK instead of Oracle's JDK is basically possible.
  However, it can be necessary to install OpenJFX explicitly.

  \subsection{Unable to create file logs/Kieker Trace Diagnosis.log}\label{LogSchreibrechte}
  If this error message appears during the start, it is very likely that \KiekerTraceDiagnosis{}  does not have write permissions in the current working directory.
  In this case the application can not write any log files, but can be used as usual.

  \section{Bugs und Featurewünsche}
  If any issues occur, which are not covered by this manual, or you find bugs or have feature wishes, you can create a ticket in our Gitlab \href{http://build.se.informatik.uni-kiel.de/gitlab/kieker/kieker-trace-diagnosis/issues/}{here}.
  In case you do not have such an account, please send a mail to the Kieker mailing list \href{https://lists.sourceforge.net/lists/listinfo/kieker-users}{here}.
  

\end{document}